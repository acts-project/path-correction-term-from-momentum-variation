\documentclass[12pt,a4paper]{scrarticle}

\usepackage{theme/acts}

\usepackage{lipsum}

\title{A Correction Term for Transport Jacobian from the Direction Variation}
\author{Beomki Yeo}


\begin{document}

\maketitle

\begin{abstract}
    %% Only use basic LaTeX markup here, it gets rendered by MathJax.

In the track extension model of ACTS, the covariance of a track is updated for every surface intersection by applying a jacobian matrix, the composition of the coordinate transformation and free space transport. In the derivation of the transport jacobian, it is necessary to consider the path difference coming from the track displacement, which appears as a direction variation term with respect to the path length. 


The motivation of this white paper is that the author failed in finding any reference explaining this clearly. (Please let the author know if anyone finds it) There are bunch of references which calculate this term with an ideal helix model, but it is not very useful with the runge-kutta based track extension model. Therefore, in this note, we are deriving the term thoroughly and demonstrate how it is implemented in the source code. 


\end{abstract}

\tableofcontents

\section{Mathematical derivation}
In track reconstruction, the exact transport of covariance matrix during the track extrapolation is important as it can largely affect the results of fitting algorithms. ACTS updates the covariance ($\mathbb{C}$) for every surface intersection by numerically integrating a jacobian ($\mathbb{J}$):
\begin{equation}
\mathbb{C}' = \mathbb{J} \times \mathbb{C} \times \mathbb{J}^T.
\end{equation}
The jacobian is the composition of sub-jacobians for coordinate transformation and free space transport. The free transport jacobian matrix is in $8 \times 8$ dimension each of which is from position vector ($\mathbf{r}$), time, unit direction ($\mathbf{t}$) vector and a charge divided by momentum ($\psi$). Its representation can be found from the variation of each parameter, where the variations of time and $\psi$ parameters can be ignored:

\begin{equation}\label{eq:dr}
d\mathbf{r} = 
    \frac{\partial \mathbf{r}}{\partial \mathbf{r_0}} \cdot d\mathbf{r_0} +
    \frac{\partial \mathbf{r}}{\partial \mathbf{t_0}} \cdot d\mathbf{t_0} + 
    \frac{\partial \mathbf{r}}{\partial \psi_0} \cdot d\psi_0 +
    \frac{\partial \mathbf{r}}{\partial s} \cdot ds,
\end{equation}

\begin{equation}
    d\mathbf{t} = 
    \frac{\partial \mathbf{t}}{\partial \mathbf{r_0}} \cdot d\mathbf{r_0} +
    \frac{\partial \mathbf{t}}{\partial \mathbf{t_0}} \cdot d\mathbf{t_0} + 
    \frac{\partial \mathbf{t}}{\partial \psi_0} \cdot d\psi_0 +
    \frac{\partial \mathbf{t}}{\partial s} \cdot ds,
\end{equation}
where $(\mathbf{r}_0, \mathbf{t}_0, \psi_0)$ and $(\mathbf{r}, \mathbf{t}, \psi)$ represent the parameters at the departure and destination surface, respectively. The last terms in each equation is caused by the path length variation due to the change in track parameters at the departure surface. (For a graphical insight, see the Fig 4.6 of Ref. \cite{tracking_book})

Our purpose is expressing the last terms as a function of the other terms to obtain the full transport jacobian. First, let us note that there is an orthogonality between surface normal vector $(\mathbf{w})$ and position variation:
 
\begin{equation}
    \mathbf{w} \cdot d\mathbf{r} = 0.
\end{equation}

Then, we can obtain $ds$ by using $\frac{\partial \mathbf{r}}{\partial s} ds = \mathbf{t}$ ds:

\begin{equation}
  ds = - \frac{1}{\mathbf{w} \cdot \mathbf{t}} \mathbf{w} \cdot 
  \left(  
  \frac{\partial \mathbf{r}}{\partial \mathbf{r_0}} \cdot d \mathbf{r_0} +  
  \frac{\partial \mathbf{r}}{\partial \mathbf{t_0}} \cdot d \mathbf{t_0} + 
  \frac{\partial \mathbf{r}}{\partial \psi_0} \cdot d \psi_0 
  \right).
\end{equation}

This can be directly put into the eq. (\ref{eq:dr}):

\begin{equation}\label{eq:dr_final}
    d\mathbf{r} = 
    \left( 
    \mathbf{I} + \mathbf{M}_r 
    \right)   
     \left(  
    \frac{\partial \mathbf{r}}{\partial \mathbf{r_0}} \cdot d \mathbf{r_0} +  
    \frac{\partial \mathbf{r}}{\partial \mathbf{t_0}} \cdot d \mathbf{t_0} + 
    \frac{\partial \mathbf{r}}{\partial \psi_0} \cdot d \psi_0 
    \right),
\end{equation}

where

\begin{equation}
    \mathbf{M}_r = - \frac{1}{\mathbf{w} \cdot \mathbf{t}} \mathbf{t} \mathbf{w}^T.
\end{equation}


We can follow the same step for $d\mathbf{t}$.  $\frac{\partial \mathbf{t}}{\partial s}$ is represented as the following:

\begin{equation}
 \frac{\partial \mathbf{t}}{\partial s} \cdot ds = \alpha K \mathbf{n} \cdot ds,
\end{equation}

where $ \alpha = \frac { | \mathbf{b} \times \mathbf {t} | } { { |\mathbf{b}| } } $, $ K = - \psi |\mathbf{b}| $ and $ \mathbf{n} = \frac {  \mathbf{b} \times \mathbf {t}  } { { |  \mathbf{b} \times \mathbf {t} | } } $ (again where $\mathbf{b}$ is the magnetic field at the destination surface.). Now it is trivial to show the following:

\begin{equation}\label{eq:dt_final}
    d \mathbf{t} = 
    \left( 
    \frac{\partial \mathbf{t}}{\partial \mathbf{r_0}}  + \mathbf{M}_t  \frac{\partial \mathbf{r}}{\partial \mathbf{r}_0}
    \right) d\mathbf{r_0} +
    \left( 
    \frac{\partial \mathbf{t}}{\partial \mathbf{t_0}} + \mathbf{M}_t \frac{\partial \mathbf{r}}{\partial \mathbf{t}_0}
    \right) d\mathbf{t_0} +
    \left( 
    \frac{\partial \mathbf{t}}{\partial \mathbf{\psi_0}} + \mathbf{M}_t \frac{\partial \mathbf{r}}{\partial \psi_0}
    \right) d\mathbf{\psi_0},
\end{equation}

where

\begin{equation}
 \mathbf{M}_t = -\frac{\psi}{\mathbf{w} \cdot \mathbf{t}} ( \mathbf{t} \times \mathbf{b}) \mathbf{w}^T .
\end{equation}

Eq. (\ref{eq:dr_final}) and (\ref{eq:dt_final}) lead to the final form of the full transport jacobian ($\mathbb{J}_{\textrm{transport}}$) as the multiplication of two matrices (Please note that the time dimension will be fully ignored after this point, thus the transport jacobian matrix will be in $7 \times 7$ dimension):

\begin{equation}\label{eq:jacobi}
    \mathbb{J}_{\textrm{transport}} = 
    \begin{pmatrix}
     \mathbf{I} + \mathbf{M}_r & 0 & 0 \\
     \mathbf{M}_t & \mathbf{I} & 0 \\
     0 & 0 & 1
    \end{pmatrix} \times
    \begin{pmatrix}
     \frac{\partial{\mathbf{r}}}{\partial{\mathbf{r_0}}} & \frac{\partial{\mathbf{r}}}{\partial{\mathbf{t_0}}} & \frac{\partial{\mathbf{r}}}{\partial{\mathbf{\psi_0}}} \\
     \frac{\partial{\mathbf{t}}}{\partial{\mathbf{r_0}}} & \frac{\partial{\mathbf{t}}}{\partial{\mathbf{t_0}}} & \frac{\partial{\mathbf{t}}}{\partial{\mathbf{\psi_0}}} \\
     0 & 0 & 1
    \end{pmatrix}    ,
\end{equation}

The first term of is the correction term and the second one is a transport jacobian calculated by Runge-Kutta integration.

\section{Implementation in ACTS}

The correction term of eq. (\ref{eq:jacobi}) can be rewritten as the following:
\begin{equation}\label{eq:correction_term}
    \begin{pmatrix}
     \mathbf{I} + \mathbf{M}_r & 0 & 0 \\
     \mathbf{M}_t & \mathbf{I} & 0 \\
     0 & 0 & 1
    \end{pmatrix} = 
    \mathbb{I} +
    \begin{pmatrix}
     t_x \\
     t_y \\
     t_z \\
     k_x \\
     k_y \\
     k_z \\
     0
    \end{pmatrix} \times
    \begin{pmatrix}
       w'_x & w'_y & w'_z & 0 & 0 & 0 & 0
    \end{pmatrix},
\end{equation}
where 
\begin{eqnarray}
  \mathbf{t} & = & (t_x ,t_y, t_z),  \\
  - \frac{\mathbf{w}}{\mathbf{w} \cdot \mathbf{t}} & = & (w'_x ,w'_y, w'_z), \label{eq:normal}  \\
  \psi(\mathbf{t} \times \mathbf{b}) & = & (k_x, k_y, k_z). \label{eq:kvec}
\end{eqnarray}

The implementation of eq. (\ref{eq:correction_term}) can be found in \verb|boundToBoundTransportJacobian| function of \verb|JacobianEngine.cpp|. Eq. (\ref{eq:normal}) can be calculated from surface classes with a track parameter input. Eq. (\ref{eq:kvec}) is the normalized equation of motion that can be retrieved from the runge-kutta stepper formalism. ACTS takes the normalized vector defined at the end of the last step of the 4th order runge-kutta stepper.

\printbibliography{}

\end{document}
