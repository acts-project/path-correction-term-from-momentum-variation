\documentclass[12pt,a4paper]{scrarticle}

\usepackage{theme/acts}

\usepackage{lipsum}

\title{A correction term for transport jacobian from the direction variation}
\author{Beomki Yeo}


\begin{document}

\maketitle

\begin{abstract}
    %% Only use basic LaTeX markup here, it gets rendered by MathJax.

This is a whitepaper example. It contains a number of example
patterns, layouts etc.
Simple math like $a + b = c$ or even $\sqrt{s} = 14$ TeV is supported!

Quisque ullamcorper placerat ipsum. Cras nibh. Morbi vel justo vitae lacus
tincidunt ultrices. Lorem ipsum dolor sit amet, consectetuer adipiscing elit. In hac
habitasse platea dictumst. Integer tempus convallis augue. Etiam facilisis. Nunc
elementum fermentum wisi. Aenean placerat. Ut imperdiet, enim sed gravida
sollicitudin, felis odio placerat quam, ac pulvinar elit purus eget enim. Nunc vitae
tortor. Proin tempus nibh sit amet nisl. Vivamus quis tortor vitae risus porta
vehicula.
\end{abstract}

\tableofcontents

\section{Introduction}

We closely follow the description in \cite{adams1995}.

\begin{figure}[ht]
    \centering
    \includegraphics[width=0.7\textwidth]{example-image-a}
    \caption{This is a caption for a figure!}
    \label{fig:example_a}
\end{figure}

\lipsum[1] 

As discussed in \autoref{fig:example_a}, the results are very promising.

\section{Method}

\begin{figure}[ht]
    \begin{subfigure}{0.5\textwidth}
        \centering
        \includegraphics[width=\textwidth]{example-image-a}
        \caption{This is a caption for a figure!}
        \label{fig:sub:example_a}
    \end{subfigure}
    \begin{subfigure}{0.5\textwidth}
        \centering
        \includegraphics[width=\textwidth]{example-image-b}
        \caption{This is a caption for a figure!}
        \label{fig:sub:example_b}
    \end{subfigure}

    \caption{This is a combined caption, \subref{fig:sub:example_a} shows one
    thing, \subref{fig:sub:example_b} shows another thing}
    \label{fig:sub}
\end{figure}

\lipsum[1] 

See \autoref{fig:sub} for an example of a combined figure.

\lipsum[1] 

\begin{equation}
    a^2 + b^2 = c^2
    \label{eq:pythagoras}
\end{equation}

The business with the triangles, our boy Pythagoras has already figured out, according to \autoref{eq:pythagoras}.

\section{Conclusion}

\lipsum[1] 

\appendix

\section{Auxiliary material}
\subsection{First subsection}

\lipsum[1] 

\begin{figure}[ht]
    \centering
    \includegraphics[width=0.7\textwidth]{example-image-a}
    \caption{This is a caption for a figure!}
    \label{fig:appendix_example_a}
\end{figure}

\printbibliography{}

\end{document}
